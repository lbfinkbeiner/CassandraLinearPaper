\documentclass[a4paper,11pt]{article}
\usepackage{jcappub} % for details on the use of the package, please see the JINST-author-manual
%\usepackage{lineno}

% Bonus packages just for Lukas
\usepackage{cprotect}
\usepackage{xcolor}
\usepackage{verbatim}

% This code formatting sucks but I don't want to output extra spaces.
\newcommand{\cbib}[1]
{\IfFileExists{biblatex.sty}
{\citep{#1}}
{[citation ``#1'' cannot be linked in the current environment]}}

% If you want to reveal self-comments, activate the first of the following two
% lines.
\newcommand{\selfcomment}[1]{\textcolor{orange}{#1}}
%\newcommand{\selfcomment}[1]{}

\arxivnumber{1234.56789} % Only if you have one
\title{\boldmath A title with some math: $x=1$}

% Collaborations

%% [A] If main author
%% \collaboration{\includegraphics[height=17mm]{collabroation-logo}\\[6pt]
%%  XXX collaboration}

%% or
%% [B] If "on behalf of"
%% \collaboration[c]{on behalf of XXX collaboration}


% Authors
% The "\note" macro will give a warning: "Ignoring empty anchor...", you can safely ignore it.

%% [A] simple case: 2 authors, same institution
%% \author[1]{A. Uthor\note{Corresponding author.}}
%% \author{and A. Nother Author}
%% \affiliation{Institution,\\Address, Country}

%% or, e.g.
%% [B] more complex case: 4 authors, 3 institutions, 2 footnotes
%% \author[a,b]{F. Irst,\}
%% \author[c]{S. Econd,}
%% \author[a,1]{T. Hird\note{Also at Some University.}}
%% \author[c,1]{and Fourth}
%% \affiliation[a]{Institution_1,\\Address, Country}
%% \affiliation[b]{Institution_2,\\Address, Country}
%% \affiliation[c]{Institution_3,\\Address, Country}

\author{Lukas Finkbeiner}
\author{Ariel G. S\'{a}nchez}
\author{Andrea S. Pezzotta}
\affiliation{Max Planck Institute for Extraterrestrial Physics,\\
Gießbachstraße 1, Munich, Germany}

% E-mail addresses: only for the corresponding author
\emailAdd{lfinkbei@mpe.mpg.de}

\abstract{Abstract...}

\graphicspath{{./res/}}

\begin{document}
\maketitle
\flushbottom

\section{Introduction}
\label{sec:intro}

\subsection{Evolution Mapping}

\citet{San21} proposes to divide up the full set of cosmological
parameters into the two categories introduced in section~\ref{sec: Pk_intro}: 
shape parameters $\Theta_s$ and evolution parameters $\Theta_e$.
Shape parameters affect the shape of the power spectrum. Examples include
$\omega_b$, $\omega_c$, and $n_s$.
Evolution parameters affect only its amplitude. They are so called because
their impact on the linear-theory power spectrum is 
indistinguishable from a time evolution of the power 
spectrum. This means that varying the redshift at which we evaluate the power
spectrum will affect only its amplitude. $h$, $\omega_k$ and
$\omega_\text{DE}$ are examples of evolution parameters. 

We adapt equation 13 from \citet{San21} to write:

\begin{equation}
\label{eq: evMapping_pSpectrum}
    P_L (k | z, \Theta_s, \Theta_e)
    =
    P_L (k | \Theta_s, \sigma_{12} \left( z, \Theta_s, \Theta_e \right))
\end{equation}

which we call the \textit{evolution mapping relation} for the power
spectrum.\footnote{As mentioned before, varying $z$ has the effect of varying 
an evolution parameter, so it appears on the RHS only as an argument 
to the $\sigma_{12}$ ``function.'' We write it separately from
$\Theta_e$ to
emphasize that $z$ does not describe a property of the Universe, but is
simply used as a proxy here for conformal time
elapsed since the Big Bang.} 
We stress that the evolution mapping relation does not hold when $P_L$ is
expressed in $h$ units.

%s In action: relabel some power spectra

The evolution mapping relation tells us that any two cosmologies with 
identical shape
parameters will yield identical power spectra as long we match the value of
$\sigma_{12}$. This matching can be accomplished merely by varying the 
redshift at which the power spectrum is evaluated.
Figure~\ref{fig: ev_mapping_demo} shows the results of this technique applied
to seven different cosmologies sharing shape parameters and differing only
in evolution parameters.

\begin{figure}[ht!]
  \centering
  \includegraphics[width=\textwidth]{intro/san21f2}
  \cprotect\caption[Demonstration of evolution mapping]{Power spectra matched
  	in $\sigma_{12}$ via evaluation at different redshifts. For each value of
  	$\sigma_{12}$
  	shown here, eight Aletheia models were plotted in different
  	colors. Each set of lines is tightly distributed and thereby demonstrates 
  	the excellent agreement of power spectra from $\sigma_{12}$-matched
  	cosmologies that agree in their shape parameters. This figure was
  	inspired by figure 2 from \citet{San21}. Aletheia model 0 was related in
  	table~\ref{tab: Aletheia_m0}. For details on the rest of the Aletheia
  	models, refer to the file
  	\verb|cosmologies.dat|\footnotemark
  	\, in the CL GitHub repository.}
  \label{fig: ev_mapping_demo}
\end{figure}

\footnotetext{\url{https://github.com/3276908917/Master/blob/main/cassL/cosmologies.dat}}

%s Now conclude with the significance

Evolution mapping is essential to this work because it greatly simplifies the 
parameter space over which we train our emulator. We can express the impact of
all evolution parameters through their impact on $\sigma_{12}$. Consequently, 
we need only train the emulator over the shape parameters and $\sigma_{12}$.
Simplifying the parameter space will
result in increased accuracy for our emulator because our training samples
will be efficiently distributed throughout the parameter space. By contrast,
if we explicitly trained over additional evolution parameters, the same set of
training samples would much less densely cover the parameter space.

\subsection{Emulation}
\label{sec: emulation_intro}

Boltzmann codes typically take on the order of three seconds to evaluate the
linear matter power spectrum for a given cosmological configuration. In a
modern Monte Carlo Markov Chain analysis, tens of thousands of such
configurations will need to be evaluated. Therefore, power
spectra evaluation becomes a bottleneck of the analysis.

This motivates the use of emulators, which rely on multi-dimensional 
interpolation instead of the equations of evolution. Emulators are trained 
over large numbers of power spectra either computed
by a Boltzmann solver or from an N-body simulation. Power spectra emulation 
has been shown to achieve excellent accuracy with computation times shorter
by several orders of magnitude.\footnote{Of course, the accuracy of a
well-constructed emulator will ultimately still be limited by the accuracy of 
the input power spectra. Recall in the case of CAMB that this is 0.1\%
\cbib{Seljak}.}

Emulators apply different machine
learning approaches in order to transform between cosmological
configurations and power spectra. The two most common choices are neural 
networks (NNs), which optimize the weights and biases on a set of connected 
nodes, and Gaussian process regressions (GPRs), which apply Bayes' theorem to  
iteratively tighten the variance of a distribution of random functions.

Emulators interpolate across a high-dimensional parameter space. This presents
a fundamental limitation: emulators must be built with the same parameter
space over which end users require for their analyses. If additional
parameters or wider priors are needed, then new emulators need to be trained.
Yet there is a large number of different cosmological parameters discussed in 
the modern literature. Emulator flexibility and generality is desirable, but 
comes at the price of a significantly more sample-expensive training phase in
order to maintain accuracy levels.

% Cut paragraph. It's mostly incorrect
\begin{comment}
	However, due to the
generality-accuracy tradeoff, currently available emulators tend to sample
only sample a few cosmological parameters and over restrictive ranges
\cbib{San21}. \textcolor{red}{This comment only applied to the emulators
trained over N-body simulations, right? So it would probably be inappropriate
to mention it in this paper, since we are emulating over Boltzmann solver
spectra}.

	``Due to the high computational cost of the required
	simulations, [...] current emulators leave out parameters such as the
	curvature of the Universe or dynamic energy models beyond the standard CPL
	parametrization'' (\cbib{San21}).
\end{comment}

The effectiveness of emulation has given rise to considerable popularity, and
several emulators of linear and nonlinear power spectra have been put 
forth in the literature.\footnote{Linear and nonlinear power spectra have 
different
applications. Linear power spectra can be used to predict $P_\text{NL}(k)$, 
such as in perturbation theory approaches. Furthermore, most non-linear 
emulators actually predict the ratio of the non-linear power spectrum to its 
linear counterpart, so $P_\text{L}(k)$ is still necessary to evaluate
$P_\text{NL}(k)$.}
\verb|baccoemu| includes both nonlinear and linear power spectrum emulators 
based on NNs. The quoted accuracies in the linear case are
0.2\% for redshifts $z \lesssim 3$ and
0.5\% for redshifts $z \lesssim 9$. This emulator is trained over 216000
\verb|CLASS|
models in 600 $k$-bins $10^{-4} \leq k$ [$h$ Mpc$^{-1}$] $\leq 50$ and 
includes massive neutrinos \cbib{Arico}. \verb|CosmoPower| also uses an NN
to emulate both linear and nonlinear power spectra. It was trained over
180,000 spectra in 420 $k$-bins over the ranges
$10^{-5} \leq k $ Mpc$^{-1} \leq 10$ and 
$0 \leq z \leq 5$. The authors have verified
its accuracy on a full inference pipeline, demonstrating that the results
are in excellent agreement regardless of the choice between \verb|CAMB| and
\verb|CLASS| for the calculation of the training spectra. In the linear case,
\citet{Mancini} quote errors of less than 0.1\%. 

\verb|COMET| is of particular interest to this work because it has already
successfully demonstrated the application of evolution mapping to the task of
linear power spectrum emulation. \verb|COMET| uses GPR, although the 
development team
has recently decided to switch to an NN \cbib{Eggemeier}.

\selfcomment{I need material on the accuracy and ranges over which
COMET can operate.}

% Other things we could talk about: make more explicit that we are using a
% large # Boltzmann / N-body power spectra in order to train the emulator in
% the first place. This of course means that the accuracy will eventually be
% limited by the accuracy of the input spectra...

% This section really is short. Am I forgetting to talk about anything?

In this work, we apply a GPR to emulate power spectra. A Gaussian Process 
(GP) is a Gaussian distribution over random functions,\footnote{The functions
themselves are deterministic.} which can be interpreted
as the infinite-dimensional generalization of the multivariate normal
distribution. The inference of continuous values with a GP prior
is known as Gaussian process regression, or Kriging. GPR is a
powerful non-linear multivariate interpolation tool because in addition to
allowing conventional predictions, GPR can also give us an uncertainty on the
prediction, which is not typically obtained when training an NN.

\selfcomment{I should put some equations here, to outline the basics
For example, I could refer to
\url{https://gaussianprocess.org/gpml/chapters/RW.pdf} as the COMET paper
does.}

NNs represent the most common approach in the literature. We elect to use a
GPR for this work for several reasons. Most importantly, we find it to be
a considerably simpler approach, more accessible to novices.
NNs invariably require much more complicated setup and tuning in the precise 
architecture and hyperparameter values (nodes per layer, learning rate, etc.) 
By contrast, as we explain in section~\ref{sec: train_emu}, a Gaussian
process regression is highly straightforward to set up and modify.

Furthermore, NNs generally need much larger sample sizes to reach comparable 
levels of accuracy. Due to various alterations in the CL code 
over its development, several regenerations of the various emulator data sets 
were necessary. This practical constraint motivated the use of a GP for our
emulator.

\selfcomment{Are there other prediction approaches besides GPs and NNs? 
If so, I need to further justify WHY we’re using GPs.
GPs work best when there are few samples and a lot of 
parameters, right? But why is that so? What is the math behind that?}

For internal references use label-refs: see section~\ref{sec:intro}.
Bibliographic citations can be done with "cite": refs.~\cite{a,b,c}.
When possible, align equations on the equal sign. The package
\texttt{amsmath} is already loaded. See \eqref{eq:x}.
\begin{equation}
\label{eq:x}
\begin{aligned}
x &= 1 \,,
\qquad
y = 2 \,,
\\
z &= 3 \,.
\end{aligned}
\end{equation}
Also, watch out for the punctuation at the end of the equations.


If you want some equations without the tag (number), please use the available
starred-environments. For example:
\begin{equation*}
x = 1
\end{equation*}

\section{Figures and tables}

All figures and tables should be referenced in the text and should be
placed on the page where they are first cited or in
subsequent pages. Positioning them in the source file
after the paragraph where you first reference them usually yield good
results. See figure~\ref{fig:i} and table~\ref{tab:i} for layout examples. 
Please note that a caption is mandatory  and it must be placed at the bottom of both figures and tables.

\begin{figure}[htbp]
\centering
\includegraphics[width=.4\textwidth]{example-image-a}
\qquad
\includegraphics[width=.4\textwidth]{example-image-b}
\caption{Always give a caption.\label{fig:i}}
\end{figure}

\begin{table}[htbp]
\centering
\begin{tabular}{lr|c}
\hline
x&y&x and y\\
\hline
a & b & a and b\\
1 & 2 & 1 and 2\\
$\alpha$ & $\beta$ & $\alpha$ and $\beta$\\
\hline
\end{tabular}
\caption{We prefer to have top and bottom borders around the tables.\label{tab:i}}
\end{table}

We discourage the use of inline figures (e.g. \texttt{wrapfigure}), as they may be
difficult to position if the page layout changes.

We suggest not to abbreviate: ``section'', ``appendix'', ``figure''
and ``table'', but ``eq.'' and ``ref.'' are welcome. Also, please do
not use \texttt{\textbackslash emph} or \texttt{\textbackslash it} for
latin abbreviaitons: i.e., et al., e.g., vs., etc.


\paragraph{Up to paragraphs.} We find that having more levels usually
reduces the clarity of the article. Also, we strongly discourage the
use of non-numbered sections (e.g.~\texttt{\textbackslash
  subsubsection*}).  Please also consider the use of
``\texttt{\textbackslash texorpdfstring\{\}\{\}}'' to avoid warnings
from the \texttt{hyperref} package when you have math in the section titles.



\appendix
\section{Some title}
Please always give a title also for appendices.





\acknowledgments

This is the most common positions for acknowledgments. A macro is
available to maintain the same layout and spelling of the heading.

\paragraph{Note added.} This is also a good position for notes added
after the paper has been written.


% Bibliography

%% [A] Recommended: using JHEP.bst file
%% \bibliographystyle{JHEP}
%% \bibliography{biblio.bib}

%% or
%% [B] Manual formatting (see below)
%% (i) We suggest to always provide author, title and journal data or doi:
%% in short all the informations that clearly identify a document.
%% (ii) please avoid comments such as "For a review'', "For some examples",
%% "and references therein" or move them in the text. In general, please leave only references in the bibliography and move all
%% accessory text in footnotes.
%% (iii) Also, please have only one work for each \bibitem.

\bibliographystyle{JHEP}
\bibliography{biblo}

\end{document}
